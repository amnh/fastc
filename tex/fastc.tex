\documentclass[12pt]{article}
\usepackage{setspace}
\usepackage{graphicx}
\usepackage{subfigure}
\usepackage{lscape}
\usepackage{flafter}  % Don't place floats before their definition
\usepackage{bm}  % Define \bm{} to use bold math fonts
\usepackage{amsmath}
\usepackage{amsfonts}
\usepackage{amssymb}
%\usepackage{MnSymbol}
\usepackage{mathabx}
\usepackage{url}
\usepackage{natbib}
\usepackage{syntax, etoolbox}
\usepackage{tipa}
\usepackage{array, multirow}
\usepackage{fullpage}
\usepackage{textcomp}

\bibliographystyle{cbe}
%\citestyle{aa}

\begin{document}

%\title{Words as Sequences in Uto-Aztecan Language Evolution and Biogeography}
\title{FASTC: A file format for multi-character sequence data}
\author{Ward C. Wheeler and Alex Washburn \\
		Division of  Invertebrate Zoology\\
		American Museum of Natural History\\
		200 Central Park West\\
		New York, NY 10024-5192\\
		USA\\
		wheeler@amnh.org\\
		212-769-5754}

\maketitle
\begin{center}
Running Title: FASTC file format
\end{center}
\newpage
\doublespacing

\section{Abstract}
Here, we define a sequence file format that allows for multi-character elements (FASTC).  The format is derived from the FASTA format of \cite{LipmanandPearson1985} and the custom alphabet format of POY4/5 \citep{POY4,Wheeleretal2015}.  The format is more general than either of these formats and can represent a broad variety of sequence-type data.

\newpage
\tableofcontents
\newpage

\section{Introduction and Motivation}
The FASTA (or Pearson) sequence format was first articulated as a component of sequence database searching tools
\citep{LipmanandPearson1985}.  This format is nearly universally recognized and used for sequence data input 
by a large variety of sequence analysis software packages.

The format is admirably simple (defined in FASTA program documentation, see \url{www.cse.unsw.edu.au/~binftools/birch/birchhomedir/doc/fasta/fasta20.doc}) with 
only two control characters (`$>$' and `;').  The semicolon (`;') denotes a comment (to end of line) and `$>$' labeling a sequence.  
The sequence label line begins with `$>$' and continues until white space or end of line is encountered.  Sequences consist of single-character
IUPAC protein and nucleic acids codes.  All other symbols are ignored (such as numbers and white space).  Line length is limited to 120 characters.

An example, valid file could look like this:\\
\texttt{;This is an example file}\\
\texttt{$>$First\_DNA sequence}\\
\texttt{ACGTTT @GGA;This is a comment}\\
\texttt{$>$Second\_DNA sequence}\\
\texttt{1 GT-A 4 TTCA }\\
This would result in the input of two sequences: First\_DNA  ACGTTTGGA and Second\_DNA GTATTCA.

POY4 \citep {POY4} and POY5 \citep {POY5, Wheeleretal2015} extended the legal character symbol set to include `-' to represent alignment gaps (an IUPAC symbol for nucleotides but not protein sequences), `X' to represent any nucleotide (in addition to `N' for nucleotides), and `?' to represent `X' or `-' (unknown element or gap).  

\subsection{Multi-character alphabets}
Situations arise where multi-character sequence elements are required or at least convenient.  These can include gene synteny, developmental, and comparative linguistic data (explained further in \citealp{SchulmeisterandWheeler2004,Wheeler2007, Wheeler2012, WheelerandWhiteley2015}).  POY4 and POY5 contain the `custom alphabet' sequence type from which the FASTC format described here is derived.

The custom alphabet file format allows for multi-character element representations (e.g. alpha, beta, gamma), but these must be prefix-free.  This allows sequence parsing to proceed more easily, but has the limitation that it requires a prefix-free alphabet and not all are.  Furthermore, it
can lead to less that easily human-legible data files such as:\\
\texttt{$>$First\_sequence}\\
\texttt{alphabetagammadelta}\\
\texttt{$>$Second\_sequence}\\
\texttt{betagammaalphadelta}\\
The custom alphabet elements can also be preceded by a tilda (`$\sim$') denoting reverse orientation (useful for gene synteny data).

\section{FASTC}
The FASTC format grows out of the FASTA and custom alphabet formats by adding a mandatory white space between elements and allowing for non-prefix free multi-character 
sequence element specification.
\pagebreak

\subsection{Grammar specification}
The file grammar is specified as follows:

\AtBeginEnvironment{grammar}{\small}
%\setlength{\grammarparsep}{12pt plus 1pt minus 1pt} % increase separation between rules
\setlength{\grammarindent}{10em} % increase separation between LHS/RHS

\begin{grammar}

  <FILE>         ::=  <SPACEMAYBE> $\,\,\Big($ `>' <IDENTIFIER> <ID_END> <SEQUENCE> $\Big)^{\Asterisk}$

  <IDENTIFIER>   ::=  <SPACE_PAD> $\Big($ <VALID_CHAR> $\Big)^{+}$ <SPACE_PAD>

  <ID_END>       ::=  `\\n' <SPACEMAYBE>
                 \alt <COMMENT> <SPACEMAYBE>

  <COMMENT>      ::=  `;' $\,\,\Big($ <INLINE_CHAR> $\Big)^{\Asterisk}$ `\\n'

  <SEQUENCE>     ::=  <ELEMENT> $\Big($ <WHITESPACE> <ELEMENT> $\Big)^{\Asterisk}$ <MAYBESPACE>
  
  <ELEMENT>      ::=  <SYMBOL>
                 \alt `[' <MAYBESPACE> <SYMBOL> <SYMBOL_LIST> <MAYBESPACE> `]'
 
  <SYMBOL>       ::=  $\Big($ <VALID_CHAR> $\Big)^{+}$
                 
  <SYMBOL_LIST>  ::=  $\Big($ <WHITESPACE> <SYMBOL> $\Big)^{\Asterisk}$
  
  <MAYBESPACE>   ::=  $\Big($ <SPACING> $\Big)^{\Asterisk}$

  <WHITESPACE>   ::=  $\Big($ <SPACING>  $\Big)^{+}$

  <SPACING>      ::=  \verb![\s]!$^{+}$  $\quad\,\quad\quad\quad \lhd \,\, \textbf{one or more spaces}$
                 \alt <COMMENT>

  <SPACE_PAD>    ::=  $\Big($ <SPACE_CHAR> $\Big)^{\Asterisk}$

  <SPACE_CHAR>   ::=  \verb![^\n\S]!     $\quad\,\quad\quad      \lhd \,\, \textbf{not a new-line or a non-space}$

  <INLINE_CHAR>  ::=  \verb![^\n]!       $\quad\,\quad\quad\quad \lhd \,\, \textbf{not a new-line}$

  <VALID_CHAR>   ::=  \verb![^;>\[\]\s]! $\quad                  \lhd \,\, \textbf{not}$ `;',$\,$ `>',$\,$ `[',$\,$ `]', $\,\textbf{or a space}$


  
\end{grammar}

It is worth noting that parentheses, '(' and '), \textit{are allowed} in an identifier but may cause conflict with with E/Newick tree file format \citep{Cardonaetal2008} containing the same identifier that has not been properly quoted. \\

Multiple identifier lines without sequence data  are not permitted:\\
\texttt{	
>foo\\
>bar\\
}

\subsection{Example implementation}

An example Haskell implementation of a FASTC file parser conforming to the grammar above can be found here:

https://github.com/amnh/fastc


\subsection{Example files}
Traditional sequence files can be represented in fastc with the inclusion of separating spaces (from \citealp{Wheeler1998c}).
\singlespace
\begin{verbatim}
>Americhernus      
T C G A G C C T C C A A T G A T A C G T T G A A A G G C G T T T A T C G T T 
G G G G C C G A C A G - - C G T C G T G G G C T C G G T T G G C C T T A 
A A A A G C T G A T C G G G T T C T C C G G C A A T T T T A C T T T G A A A A
  A A T T A G G G T G C T C A A G T G C C 
>Chanbria;From Genbank
T C T A G A C T G G T G G T C C G C C T C T G G T G G T T A C T A C C T G 
G C C T A A A C A A T T T G C C G G T T T T C C C T T G G T G C T C T T C A 
C C G A G T G T C T T G G G G G A C T G G T A C G T T T A C T T T G A A G A 
A A C T A G A G T G C T C A A A - C A G G C G T A A C
G C C 
>Gea
 T C C G G C C G G A C G G G T C C G C C T A C C G G T G G T 
T A C T G T T C G C T G C C G A G C T T C A G G G G G C C G C T G T C G 
 A T G A T C T T C A T C G G T T A T C T T C C G T A A C C C T C A C 
G T T T A C T T T G A A A A A A T T A G A G T G C T C A A A G C A G C - -
G C G A C G C C 
>Hypochilus;An interesting spider
- T C C A G A C G G G C G G T C C G C C T A A C G G T G G T T A C T G C C T 
G G C C T G A A C A A C C A G C C G G T T T C C C T A G A T G A T 
C T T C A T T G A T T G T C T T G G G T G A C C G G C A C G T T T A C T T
T G A A A A A A T T A G A G T G C T C A A A G C A G C G T G A C G C C 
 \end{verbatim}
 \doublespace
Linguistic data based on the international phonetic alphabet (IPA) may contain multi-character sequence elements
(from \citealp{Whiteleyetal2019}).
\singlespace
\begin{verbatim}
>Ngombe
\ve b \'O
>Mbesa
\'{\i} f \'{\i} n j \`{\i}
>Likile
b o s \'a m b \'a
>Mongo
l O w \'O
\end{verbatim}

Gene synteny data can also be represented.
\begin{verbatim}
>species_1
CYTB NAD1 12SrDNA 16SrDNA
>species_2
CYTB 12SrDNA 16SrDNA
>species_3
16SrDNA 12SrDNA NAD1 CYTB
\end{verbatim}

The following file would be invalid due to absence of data for Mbesa.

\begin{verbatim}
>Ngombe
e b \'O
>Mbesa
>Likile
b o s \'a m b \'a
>Mongo
l O w \'O
\end{verbatim}
\doublespace
\section{Summary}
The fastc format naturally generalizes existing sequence format files and can be employed to
represent a diversity of
data types with linear ordering.

\section{Acknowledgements}
This work was supported by DARPA SIMPLEX (``Integrating Linguistic, Ethnographic, and Genetic Information of Human Populations: Databases and Tools,'' DARPA-BAA-14-59 SIMPLEX TA-2, 2015-2018). 
\newpage
\bibliography{big-refs-3}
\end{document}




\grid
\grid
